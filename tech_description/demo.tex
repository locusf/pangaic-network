\documentclass[10pt]{beamer}

\usetheme[progressbar=frametitle]{metropolis}
\usepackage{appendixnumberbeamer}

\usepackage{booktabs}
\usepackage[scale=2]{ccicons}
\usepackage{listings}
\usepackage{xspace}
\newcommand{\themename}{\textbf{\textsc{metropolis}}\xspace}

\title{Pangaeic network}
\subtitle{Technical theory description}
\date{\today}

\author{Aleksi Suomalainen}

% \titlegraphic{\hfill\includegraphics[height=1.5cm]{logo.pdf}}

\begin{document}

\maketitle

\begin{frame}{Table of contents}
  \setbeamertemplate{section in toc}[sections numbered]
  \tableofcontents[hideallsubsections]
\end{frame}

\section{Introduction}

\begin{frame}{Pangaeic network}

\end{frame}

\begin{frame}{Machine languages}
Machine languages, thus computers, can be abstracted as 2 main types: CISC (Complex instruction set computer), RISC (Reduced instruction set computer). EVM falls in to the category RISC, so any language has a subset of its instructions can be used to almost directly translate between its instruction set and EVM instruction set.
\begin{columns}[T,onlytextwidth]
	\column{0.5\textwidth}
      \begin{alertblock}{CISC}
         xor     eax, eax\\
        mov     dword ptr [rbp - 4], 0\\
        pop     rbp\\
        ret
      \end{alertblock}
    \column{0.5\textwidth}
	\begin{exampleblock}{RISC}
         .pad    \#4\\
        sub     sp, sp, \#4\\
        mov     r0, \#0\\
        str     r0, [sp]\\
        add     sp, sp, \#4\\
        bx      lr
      	\end{exampleblock}
\end{columns}
\end{frame}


\section{Blockchains}

\begin{frame}{Ethereum}
\end{frame}

\begin{frame}{Storj}
\end{frame}


\section{Physical access}

\begin{frame}{New physical medium possibilities}
  
\end{frame}

\section{Conclusion}

\begin{frame}{Summary}
\end{frame}

\begin{frame}[standout]
  Questions?
\end{frame}

\end{document}
